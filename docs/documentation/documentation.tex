\documentclass{article}

\title{PasswordGenerator.py \\ \large Documentazione}
\author{}
\date{}

\usepackage{lipsum}
\usepackage{outlines}
\usepackage{xparse}
\usepackage{xcolor}
\usepackage{listings}

\lstdefinestyle{smallcode}{
    basicstyle=\small\ttfamily,
    columns=fullflexible,
    keepspaces=true,
    xleftmargin=0.5em,
    xrightmargin=0.5em,
}
    
\begin{document}
    
    \maketitle

    \section*{Panoramica}
        La classe \texttt{PasswordGenerator} fornisce funzionalità per generare password sicure basate su set di caratteri definiti dall'utente. Include anche metodi per calcolare la complessità, l'entropia, la forza della password generata e il tempo necessario per violarla con un attacco brute force.

    \section*{Attributi}
        \begin{outline}
            \1 \texttt{lowercase}: Lista di lettere minuscole
            \1 \texttt{uppercase}: Lista di lettere maiuscole
            \1 \texttt{digits}: Lista di cifre
            \1 \texttt{specialChars}: Lista di caratteri speciali
            \1 \texttt{charsAvailable}: Lista di caratteri disponibili per la generazione della password basata sulle scelte dell'utente
            \1 \texttt{password}: La password generata come stringa
        \end{outline}

    \section*{Metodi}
        \subsection*{\texttt{\_\_init\_\_(self)}}
            Inizializza gli attributi della classe \texttt{PasswordGenerator}.
        
        \subsection*{\texttt{buildCharsAvailable(self, choices)}}
            Costruisce una lista di caratteri disponibili per la generazione della password basata sulle scelte dell'utente.
            \begin{outline}
                \1 \textbf{Parametri}:
                    \2 \texttt{choices} (lista di int): Lista contenente le scelte dei tipi di carattere (1 per lettere minuscole, 2 per lettere maiuscole, 3 per cifre, 4 per caratteri speciali).
                \1 \textbf{Restituisce}:
                    \2 \texttt{charsAvailable} (lista di str): Lista di caratteri disponibili per la generazione della password.
            \end{outline}
        
        \subsection*{\texttt{buildPassword(self, length, charsAvailable)}}
            Genera una password della lunghezza specificata utilizzando i caratteri dal set disponibile.
            \begin{outline}
                \1 \textbf{Parametri}:
                    \2 \texttt{length} (int): La lunghezza desiderata della password.
                    \2 \texttt{charsAvailable} (lista di str): Lista di caratteri disponibili per la generazione della password.
                \1 \textbf{Restituisce}:
                    \2 \texttt{password} (str): La password generata.
            \end{outline}

        \subsection*{\texttt{userInput(self, charTypes)}}
            Converte l'input dell'utente per le scelte dei set di caratteri in una lista di interi.
            \begin{outline}
                \1 \textbf{Parametri}:
                    \2 \texttt{charTypes} (lista di str): Lista delle scelte dei set di caratteri inserite dall'utente.
                \1 \textbf{Restituisce}:
                    \2 \texttt{newList} (lista di int): Lista delle scelte valide e non duplicate dei set di caratteri.
            \end{outline}

        \subsection*{\texttt{pwStrength(self, password)}}
            Calcola l'entropia (forza) della password generata.
            \begin{outline}
                \1 \textbf{Parametri}:
                    \2 \texttt{password} (str): La password generata.
                \1 \textbf{Restituisce}:
                    \2 \texttt{entropy} (float): L'entropia della password in bit
            \end{outline}

        \subsection*{\texttt{calcViolation(self, entropy)}}
            Calcola il numero di tentativi e il tempo necessario per violare la password generata utilizzando un attacco brute force.
            \begin{outline}
                \1 \textbf{Parametri}:
                    \2 \texttt{entropy} (float): L'entropia della password in bit
                \1 \textbf{Restituisce}:
                    \2 \texttt{violationInfo} (str): Informazioni sul numero di tentativi e il tempo necessario per un attacco brute force.
            \end{outline}
    
    \section*{Esempio di utilizzo}
        \begin{lstlisting}[style=smallcode]
    import random
    import math

    # Istanziare la classe PasswordGenerator
    password_gen = PasswordGenerator()

    # Input dell'utente per le scelte dei set di caratteri:
    choices = password_gen.userInput(['1', '2', '3', '4'])

    # Costruisce la lista di caratteri disponibili per generare la password
    chars_available = password_gen.buildCharsAvailable(choices)

    # Genera una password di lunghezza 12
    password = password_gen.buildPassword(12, chars_available)
    print(f"Password Generata: {password}")

    # Calcola l'entropia della password generata
    entropy = password_gen.pwStrength(password)
    print(f"Entropia della Password: {entropy} bit")

    # Calcola il tempo necessario per violare la password
    violation_info = password_gen.calcViolation(entropy)
    print(violation_info)
        \end{lstlisting}

\end{document}
\documentclass{scrartcl}

\title{Generatore Password Sicure}
\subtitle{Relazione di progetto}
\author{Mariottini, Ramazzini, Florea, Trandela, Cont, Nazih}
\date{}

\usepackage{lipsum}
\usepackage{xparse}
\usepackage{xcolor}
% \usepackage[margin=2cm]{geometry} su tutti i lati

\NewDocumentCommand{\codeword}{v}{
    \texttt{\textcolor{blue}{#1}}
}

\begin{document}

    \maketitle

    \section{Introduzione}

        \subsection{Descrizione}

            Un progetto Python generare password casuali sicure. Le password vengono create combinando lettere maiuscole, minuscole, numeri e caratteri speciali. Il programma offre funzionalità di personalizzazione, come la scelta della lunghezza e dei tipi di caratteri da includere, e fornisce una valutazione della forza delle password generate.

    \section{Guida al progetto}

        \subsection{Tecnologie utilizzate}

            \begin{itemize}            
                \item Python 3.x
                \item Tkinter (per l'interfaccia grafica)
                \item GitHub (per la gestione del progetto e il versionamento)
            \end{itemize}

        \subsection{Struttura del progetto}

            \begin{itemize}
                \item \codeword{/src} codice sorgente
                \item \codeword{/docs} documentazione e relazione finale
                \item \codeword{/drafts} bozze e idee per la realizzazione del progetto
            \end{itemize}

    \section{Diario}

        \subsection{Day1 - 04/12/2024}
            \begin{itemize}
                \item Brainstorming idee realizzazione progetto
                \item Divisione ruoli
            \end{itemize}
        
        \subsection{Day2 - 06/12/2024}
            \begin{itemize}
                \item Funzione per creare la password [Mariottini, Ramazzini, Trandela]
                \item Funzione per gestire l'input dell'utente [Nazih, Florea, Cont]
            \end{itemize}
            
        \subsection{Day3 - 09/12/2024}
            \begin{itemize}
                \item Funzione per calcolare la forza della password
                \item Funzione per calcolare il tempo necessario per violare la password
            \end{itemize}

        \subsection{Day4 - 19/12/2024}
        \begin{itemize}
            \item Funzione per calcolare la complessità della password
            \item Creazione GUI
        \end{itemize}

\end{document}
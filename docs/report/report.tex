\documentclass{scrartcl}

\title{Generatore Password Sicure}
\subtitle{Relazione di progetto}
\author{Mariottini, Ramazzini, Florea, Trandela, Cont, Nazih}
\date{}

\usepackage{lipsum}
\usepackage{outlines}
\usepackage{xparse}
\usepackage{xcolor}
% \usepackage[margin=2cm]{geometry} su tutti i lati

\NewDocumentCommand{\codeword}{v}{
    \texttt{\textcolor{blue}{#1}}
}

\begin{document}

    \maketitle

    \section{Introduzione}

        \subsection{Descrizione}

            Un progetto Python generare password casuali sicure. Le password vengono create combinando lettere maiuscole, minuscole, numeri e caratteri speciali. Il programma offre funzionalità di personalizzazione, come la scelta della lunghezza e dei tipi di caratteri da includere, e fornisce una valutazione della forza delle password generate.

    \section{Guida al progetto}

        \subsection{Tecnologie utilizzate}

            \begin{outline}            
                \1 Python 3.x
                \1 Tkinter (per l'interfaccia grafica)
                \1 GitHub (per la gestione del progetto e il versionamento)
            \end{outline}

        \subsection{Struttura del progetto}

            \begin{outline}
                \1 \codeword{/src} codice sorgente
                \1 \codeword{/docs} documentazione e relazione finale
                \1 \codeword{/drafts} bozze e idee per la realizzazione del progetto
            \end{outline}

    \section{Diario}

        \subsection{Day1 - 04/12/2024}
            \begin{outline}
                \1 Brainstorming idee realizzazione progetto
                \1 Divisione ruoli
            \end{outline}
        
        \subsection{Day2 - 06/12/2024}
            \begin{outline}
                \1 Funzione per creare la password [Mariottini, Ramazzini]
                \1 Funzione per gestire l'input dell'utente [Nazih, Florea, Cont]
            \end{outline}
            
        \subsection{Day3 - 09/12/2024}
            \begin{outline}
                \1 Funzione per calcolare la forza della password [Mariottini]
                \1 Funzione per calcolare il tempo necessario per violare la password
            \end{outline}

        \subsection*{Meanwhile}
            \begin{outline}
                \1 Matteo inizia gui
                \1 Angelo prova complessita

            \end{outline}    
        
        \subsection{Day4 - 19/12/2024}
        \begin{outline}
            \1 Funzione per calcolare la complessità della password
                \2 Lunghezza [Antonio]
                \2 Pattern [Bilal]
                \2 Varieta [Giulia]
                \2 Vocabolario [Tommaso]
            \1 GUI [Mariottini]
        \end{outline}

\end{document}
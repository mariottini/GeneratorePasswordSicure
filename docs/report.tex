\documentclass[twocolumn]{article}

\title{Generatore Password Sicure}
\author{Mariottini, Ramazzini, Florea, Trandela, Cont, Nazih}
\date{}

\usepackage{lipsum}
\usepackage{xparse}
\usepackage{xcolor}
% \usepackage[margin=2cm]{geometry} su tutti i lati

\NewDocumentCommand{\codeword}{v}{
    \texttt{\textcolor{blue}{#1}}
}

\begin{document}

\maketitle

\section{Introduzione}

    \subsection{Descrizione}

        Un progetto Python generare password casuali sicure. Le password vengono create combinando lettere maiuscole, minuscole, numeri e caratteri speciali. Il programma offre funzionalità di personalizzazione, come la scelta della lunghezza e dei tipi di caratteri da includere, e fornisce una valutazione della forza delle password generate.

\section{Guida al progetto}

    \subsection{Tecnologie utilizzate}

        \begin{itemize}            
            \item Python 3.x
            \item Tkinter (per l'interfaccia grafica)
            \item GitHub (per la gestione del progetto e il versionamento)
        \end{itemize}

    \subsection{Struttura del progetto}

        \begin{itemize}
            \item \codeword{/src} codice sorgente
            \item \codeword{/docs} documentazione e relazione finale
            \item \codeword{/drafts} bozze e idee per la realizzazione del progetto
        \end{itemize}

\section{Diario}

    \subsection{Day1 - 04/12/2024}
        \begin{itemize}
            \item Brainstorming idee realizzazione progetto
            \item Divisione ruoli
        \end{itemize}

\end{document}